\section{Multithreading mit OpenMP}
In diesem Kapitel m�chten wir eine Bibliothek vorstellen mit der in C sogenanntes ``Threads'' realisiert sind. Threads erm�glichen es, mehrere Dinge gleichzeitig zu berechnen. In alten Heimcomputern ist dies nicht m�glich -- das h�chste der Gef�hle ist hier, zwei Berechnungen sehr schnell abwechselnd durchzuf�hren. In modernen Computern oder sogar in Computersystemen, die f�r paralleles Rechnen ausgelegt sind, stehen mehrere CPUs (oder CPUs mit mehreren Kernen) zur Verf�gung. Alle C-Programme, die wir bisher geschrieben haben, nutzen maximal eine CPU bzw. maximal einen CPU-Kern.

Die Idee beim Threading ist, neben dem Hauptprogramm (ab jetzt \emph{Masterthread} genannt) weitere Nebenprogramme zu starten (ab jetzt \emph{Workerthreads} genannt). Den Masterthread und die Workerthreads zusammen nennt man das \emph{Threadteam}.

TODO: BILD

Wir wollen anhand des folgenden Beispiels zeigen, wie einfach OpenMP verwendet werden kann:

\begin{codes}[,label=code:openmp:bsp]
int main() {
    int i;
    double a[10000];
    for(i=0; i<10000; i++) {
        a[i] = i*i / 2.0;
    }
    return 0;
}
\end{codes}

In diesem einfachen Programm wird ein Array mit den halben Quadratzahlen gef�llt. Zu beachten ist, dass die Berechnung der n�chsten Quadratzahl nicht von der vorhergehenden abh�ngt, also k�nnen wir durch das Hinzuf�gen von nur zwei Zeilen die Berechnung parallelisieren:

\begin{codes}[,label=code:openmp:parallelbsp]
#include <omp.h>
int main() {
    int i;
    double a[10000];
    #pragma omp parallel for
    for(i=0; i<10000; i++) {
        a[i] = i*i / 2.0;
    }
    return 0;
}
\end{codes}

Verwendet man den gcc, muss beim  Kompilieren das Compiler-Flag �-fopenmp� angegeben werden. Es wei�t den gcc dazu an, das ``Compiler-Plugin'' �openmp� zu verwenden. Beim Erreichen von Zeile $5$ erzeugt der Masterthread einige Workerthreads und die Berechnung in der �for�-Schleife wird auf sie aufgeteilt -- diesen Vorgang nennt man \emph{Fork}. In Zeile 8 ist die Schleife zuwende und die Workerthreads verschwinden wieder -- dies nennt man \emph{Join}.

Die Anweisung �#pragma omp parallel for� ist eine Kurzschreibweise f�r andere Anweisungen, diese werden wir zun�chst kennen lernen. Solche Anweisungen von OpenMP haben die folgende Form: 

\begin{alltt}
    #pragma opm \ph{DIRECTIVE} \opt{CLAUSES}
\end{alltt}

Das weitere Ziel ist nun, Worte zu lernen, die als \ph{DIRECTIVE} und \opt{CLAUSES} angegeben werden k�nnen:

\subsection{Forking und Joining}
Um an irgendeiner Stelle zu Forken -- also den Masterthread anzuweisen Workerthreads zu erzeugen -- verwendet man die Direktive �parallel�. Danach kann ein C-Block angegeben werden, der dan von allen Threads gleicherma�en ausgef�hrt wird: 

\begin{codes}[,label=code:openmp:parallel]
#pragma omp parallel
{
    do_work();
}
\end{codes}

In diesem Beispiel wird die Funktion �do_work� von allen Threads aufgerufen. Die Anzahl der Threads kann �ber die Umgebungsvariable �OMP_NUM_THREADS� oder �ber die Klausel �thread_num(�\ph{ANZAHL}�)� gesteuert werden:

\begin{codes}[,label=code:openmp:parallel]
#pragma omp parallel thread_num(2)
{
    if (omp_get_thread_num() == 0) do_work();
    else do_other_work();
}
\end{codes}

GOON

\subsection{Schleifen}

\subsection{Sektionen}
